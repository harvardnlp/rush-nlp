%%% A template to produce a nice-looking Curriculum Vitae.
%%% Kieran Healy <kjhealy@gmail.com>
%%% Most recent version is at http://kjhealy.github.com/kjh-vita
%%%
%%% ------------------------------------------------------------------------
%%% Requirements (should be included in a modern tex distribution):
%%% ------------------------------------------------------------------------
%%% xelatex
%%% fontspec.sty
%%% hyperrref.sty
%%% xunicode.sty
%%% color.sty
%%% url.sty
%%% fancyhdr.sty
%%%
%%% ------------------------------------------------------------------------
%%% Optional
%%% ------------------------------------------------------------------------
%%% git
%%% vc.sty
%%% revnum.sty
%%% Fonts
%%%
%%% ------------------------------------------------------------------------
%%% Note
%%%------------------------------------------------------------------------
%%% Because this is a hand-tweaked file, be on the look out for \medksip,
%%% \bigskip and \newpage commands here and there, which are used to balance
%%% the layout or avoid widows & orphans, etc. You should of course add or
%%% remove these as needed.
%%%------------------------------------------------------------------------

\documentclass[10pt]{article}
\renewcommand{\arraystretch}{1.3}

%%%------------------------------------------------------------------------
%%% Metadata
%%%------------------------------------------------------------------------

%% Change as needed. Or just add me as a coauthor. Only some of these are
%% used below in the hyperref declaration and address banner section.
\def\myauthor{Alexander Rush}
\def\mytitle{Vita}
\def\mycopyright{\myauthor}
\def\mykeywords{}
\def\mybibliostyle{plain}
\def\mybibliocommand{}
\def\mysubtitle{}
\def\myaffiliation{cornell}
\def\myaddress{Computer Science Department}
\def\myemail{arush@cornell.edu}
\def\myweb{http://rush-nlp.com}
\def\myphone{(215) 317-8089}
\def\myfax{srush\_nlp}
\def\myversion{}
\def\myrevision{}


\def\myaffiliation{Cornell University}
\def\myauthor{Alexander Rush}
\date{} % not used (revision control instead)
\def\mykeywords{Rush, Alexander, Alexander Rush, Vita, CV, Resume, Computer Science}

%%%------------------------------------------------------------------------
%%% Git version tracking
%%%------------------------------------------------------------------------

%% If you don't use git or the vc package (from CTAN), comment this out.
%% If you comment it out, be sure to remove the \rfoot comment below, too.
% \input{vc}

%%%------------------------------------------------------------------------
%%% Required style files
%%%------------------------------------------------------------------------
\usepackage{url,fancyhdr}
%%\usepackage{revnum} % for reverse-numbered publications (revnumerate environment) if needed.

%% needed for xelatex to work
% \usepackage{fontspec}
% \usepackage{xunicode}

%% color for the links
\usepackage[usenames,dvipsnames]{xcolor}
\usepackage{palatino}
%% hyperlinks
\usepackage[
colorlinks=true,
	urlcolor=BlueViolet,
	plainpages=false,
  	pdfpagelabels,
  	bookmarksnumbered,
  	pdftitle={\mytitle},
  	pagebackref,
  	pdfauthor={\myauthor},
  	pdfkeywords={\mykeywords}
  	]{hyperref}

%%%------------------------------------------------------------------------
%%% Document
%%%------------------------------------------------------------------------
\begin{document}

%% Choose fonts for use with xelatex
%% Minion and Myriad are widely available, from Adobe.
%% Pragmata is available to buy at http://www.fsd.it/fonts/pragma.htm
%% and is worth every penny. Any good monospace font will work fine, though.
%% Consolas or inconsolata are good alternatives.
% \setromanfont[Mapping={tex-text},Numbers={OldStyle},Ligatures={Common}]{arial}
% \setsansfont[Mapping=tex-text,Colour=AA0000]{Myriad Pro}
% \setmonofont[Mapping=tex-text,Scale=0.9]{Inconsolata}


%%%------------------------------------------------------------------------
%%% Local commands
%%%------------------------------------------------------------------------

%% Marginal header
%% Note: as the document goes on you may need to introduce a (gradually increasing)
%% \vspace element to keep the marginal header pleasingly aligned with the first
%% item in the body text. Like this: \marginhead{ {\vskip 0.4em}Grants}, or
%% \marginhead{ {\vskip 0.8em}Service}. Experiment as needed.
\newcommand{\marginhead}[1]{\marginpar{\textsf{ {\footnotesize\vspace{-1em}\flushright #1}}}}


%% custom ampersand (font consistent with the one chosen above)
\newcommand{\amper}{ {\fontspec[Scale=.95,Colour=AA0000]{Minion Pro Medium}\selectfont\&\,}}

%% No bullets on labels
\renewcommand{\labelitemi}{~}

%% Custom hanging indent for vita items
\def\ind{\hangindent=1 true cm\hangafter=1 \noindent}
%\def\ind{\hangindent=18pt\hangafter=1 \noindent}
\def\labelitemi{~}
\renewcommand{\labelitemii}{~}

%%%------------------------------------------------------------------------
%%% Page layout
%%%------------------------------------------------------------------------
\pagestyle{fancy}
\renewcommand{\headrulewidth}{0pt}
\fancyhead{}
\fancyfoot{}
\rhead{ {\scriptsize\thepage}}

%% git revision control footer
%\rfoot{\texttt{\scriptsize \VCRevision\ on \VCDateTEX}} % git revision info inserted via external script -- see docs for vc package for details. comment out this line if you're not using vc, and also remove the \input{vc} line above.

%%%------------------------------------------------------------------------
%%% Address and contact block
%%%------------------------------------------------------------------------
\begin{minipage}[t]{2.95in}
 \flushright {\footnotesize \href{http://rush-nlp.com}{Cornell University} \\ Cornell Tech, \\ \vspace{-0.05in} New York, NY }

\end{minipage}
\hfill
%\begin{minipage}[t]{0.0in}
% dummy (needed here)
%\end{minipage}
\hfill
\begin{minipage}[t]{1.7in}
  \flushright %\footnotesize Phone: \myphone \\
  {\scriptsize  \texttt{\href{mailto:\myemail}{\myemail}}} \\
  {\scriptsize  \texttt{\href{\myweb}{\myweb}}} \\
  {\scriptsize  \texttt{\href{http://twitter.com/\myfax}{@\myfax}}} \\
\end{minipage}


\medskip

%% Name
\noindent{\Large {Alexander M. Rush}}
\reversemarginpar
\medskip


\marginhead{Appointment}
\noindent\emph{Cornell University Computer Science Department \vspace{0.01in}}\\
\ind 2019-.  Associate Professor of Computer Science (without Tenure)

\medskip
\noindent\emph{Hugging Face \vspace{0.01in}}\\
\ind 2019-.  NLP Open-Source Startup

\medskip
\noindent\emph{Harvard University School of Engineering and Applied Sciences \vspace{0.01in}}\\
\ind 2015-2019.  Assistant Professor of Computer Science

\medskip
\noindent\emph{Facebook Artificial Intelligence Research Lab \vspace{0.01in}}\\
\ind 2015.  Post-Doctoral Fellowship Advisor: Yann LeCunn

\bigskip

\marginhead{Education}

\noindent\emph{Massachusetts Institute of Technology \vspace{0.01in}}\\
\ind 2009-2014.  Ph.D, Computer Science. Advisor: Michael Collins\\
\ind Dissertation: \emph{Relaxation Methods for Natural Language Decoding}. %\vspace{-0.1in}



\medskip
\noindent\emph{Harvard University\vspace{0.02in}}\\
\ind 2007. B.A., Computer Science.

\bigskip

%% Publications

\marginhead{ {\vskip 0.4em} Awards}
\hspace{-1cm} \begin{tabular}{lp{11.5cm}}
2023 & Best Paper Runner-up -  NeurIPS \\
     & Outstanding Paper -  EMNLP \\
2021 & Best Demo Paper -  EMNLP \\
     & Outstanding Short Paper -  NAACL \\
     & Sloan Fellowship\\
2020 & Best Demo Paper (Runner-Up), ACL   \\
& Best Paper - DAC (Hardware)  \\
& Best Demo Paper - EMNLP  \\
2019 & NSF Career Award \\
& Best Demo Paper - Nominee, ACL   \\
2018 & Senior Program Chair, ICLR  \\
& Best Paper - Runner-Up, VAST (Visualization)  \\
2017 & Best Demo  - Runner-Up, ACL  \\
 & Invitation IJCAI Early Research Spotlight \\
& Best Paper - Runner-Up, EMNLP \\
2015 & NIPS Deep Learning Symposium (Invited Paper)   \\
2012 & Best Paper Award, NAACL \\
2010 & Best Paper Award, EMNLP \\
\end{tabular}


\marginhead{ {\vskip 0.4em} Grants}

\hspace{-1cm} \begin{tabular}{lp{11.5cm}}
2019 & NSF Career Award \\
& Sony Faculty Awards \\
2018& Google, Facebook, and Amazon AWS Faculty Awards \\
2017 & Bloomberg and Intel AI Collaboration Faculty Awards \\
2016 & Microsoft Azure  and Samsung AI Award \\
2015 & Google Faculty Award \\
\end{tabular}
\pagebreak


\bigskip


\marginhead{ {\vskip 0.3em}Publications}

\noindent (Full list: \url{http://bit.do/alexander-rush}) \\

\medskip
\noindent
\textbf{Highly Cited Publications (Google Scholar Metrics)}

\ind Alexander M. Rush, Sumit Chopra, and Jason Weston. \emph{\href{ http://arxiv.org/pdf/1509.00685.pdf }{ A Neural Attention Model for Abstractive Sentence Summarization.} }\emph{ EMNLP 2015. (1600 citations, 3rd Most Cited AAAI Paper 2015-2020) }

\medskip


\ind Yoon Kim, Yacine Jernite, David Sontag, and Alexander M. Rush. \emph{\href{ https://arxiv.org/pdf/1508.06615v4 }{ Character-Aware Neural Language Models.} }\emph{ AAAI 2016, (1250 citations, 3rd Most Cited AAAI Paper 2015-2020) }

\medskip

\ind Hendrik Strobelt, Sebastian Gehrmann, Hanspeter Pfister, and Alexander M. Rush. \emph{\href{ http://lstm.seas.harvard.edu/ }{ LSTMVis: A Tool for Visual Analysis of Hidden State Dynamics in Recurrent Neural Networks.} }\emph{ InfoVis 2017, (139 citations, 12th Most Cited IEEE Transactions on Visualization and Computer Graphics Paper 2015-2020) }



\medskip
\ind Guillaume Klein, Yoon Kim, Yuntian Deng, Jean Senellart, Alexander M. Rush. \emph{\href{ https://arxiv.org/abs/1701.02810 }{ OpenNMT: Open-Source Toolkit for Neural Machine Translation.} }\emph{ ACL Demo 2017 (900 citations, Best Demo Runner-up, 5th Most Cited ACL Paper 2015-2020) }

\medskip

\noindent\textbf{All Conference Papers \vspace{0.01in}}


[{{loop.index}}] \ind {{paper.authors}}. \emph{\href{ {{paper.pdf}} }{ {{paper.title}}.} }\emph{ {{paper.conference}} }

\medskip




[{{loop.index + papers|length}}] \ind {{paper.authors}}. \emph{\href{ {{paper.pdf}} }{ {{paper.title}}.} }\emph{ {{paper.conference}} }


\vspace{0.3in}

\noindent\emph{Journal Papers \vspace{0.01in}}

\ind Alexander M. Rush and Michael Collins. \emph{\href{http://www.cs.columbia.edu/~mcollins/acltutorial.pdf}{A Tutorial on Dual Decomposition and Lagrangian Relaxation for Inference in Natural Language Processing}}. Journal of Artificial Intelligence Research 2012.



% %\end{revnumerate}
% \begin{itemize} \itemsep -2pt %
% \item Helped develop optimized C++ parser and interpreter for FBML, a domain-specific markup language for applications.

% \medskip

% \item Built and designed a distributed system for tracking resource constraints of third-party applications. Awarded patent.
% \medskip

% \item Contributor on Internationalization project which utilizes crowd-sourcing to translate Facebook into 70+ languages.
% \end{itemize}

% \ind Technical Associate (Intern), {\sl Bridgewater Associates}  June -- September 2006,
%  Westport, CT. Implemented trade scheduling and equity analysis algorithms.
% \begin{itemize} \itemsep -2pt % Reduce space between items
% \item Designed and implemented large-scale trade scheduling and equity analysis algorithms.
% \end{itemize}


\marginhead{ {\vskip 0.4em}Professional Service}
\medskip


\bigskip

\ind Secretary:
\ind ICLR Board 2021-2024

\ind General  Chair:
\ind ICLR (Deep Learning) 2020

\ind Senior Program Chair:
\ind ICLR (Deep Learning) 2019

\ind Senior Area Chair:
\ind TACL 2020; NeurIPS 2019; ACL (Generation) 2019; ACL (Generation) 2020

\ind Area Chair:
\ind NAACL (Machine Learning) 2020; NIPS 2018; NAACL (Machine Learning) 2016; ICLR (NLP) 2017, 2018; EMNLP (Generation/Summarization) 2017; ACL (Parsing and Tagging) 2017

\ind Tutorial Chair: NAACL 2016

\bigskip

\marginhead{ {\vskip 0.4em} Graduate Field Activity}
\medskip

\ind Field Member - Computer Science

\ind PhD Students


\ind Graduated Harvard Students

\begin{itemize}
\item Sam Wiseman (2019); Assistant Professor, Duke University
\item Yoon Kim (2020); Assistant Professor, MIT
\item Sebastian Gehrmann (2020); Google NLP
\end{itemize}

\ind Current Harvard Students

\begin{itemize}
\item Yuntian Deng (Target 2022)
\item Jiawei Zhou (Target 2022)
\end{itemize}

\ind Current Cornell Students

\begin{itemize}
\item Justin Chiu (Target 2023)
\item Jack Morris (Admitted)
\item Woojeong Kim (Admitted)
\item Celine Lee (Admitted)
\end{itemize}

\ind Former Undergraduate Students

\begin{itemize}
\item Demi Guo (Stanford)
\item Lisa Li (Stanford)
\item Keyon Vafa (Columbia)
\item Rachit ()

\end{itemize}


\marginhead{ {\vskip 0.4em}University Service}
\medskip

\ind Program Director 2020-; Computer Science Program, the largest Masters program at Cornell Tech.

\ind Admissions Committee 2020
\bigskip


\marginhead{ {\vskip 0.3em}Teaching}

\hspace{-1cm} \begin{tabular}{lp{11.5cm}}
2021 & \ind  Instructor, Topics in Machine Learning and NLP (10 students) , Cornell Tech, Spring. \\
2020 & \ind  Instructor, Topics in Machine Learning and NLP (100 students), Cornell Tech, Spring. \\
     & \ind  Instructor,  Machine Learning Engineering (100 students) (Rated 4.75/5), Cornell Tech, Fall. \\

                2019 & \ind  Instructor, Machine Learning for NLP (50 students) , Harvard University, Spring. \\

2018 & \ind  Instructor, Machine Learning for NLP (50 students) (Rated 4.8/5) , Harvard University, Spring. \\
& \ind  Instructor, Advanced Machine Learning (100 students), Harvard University, Fall. \\
2017 & \ind  Instructor, Machine Learning (250 students), Harvard University, Spring. \\
& \ind  Instructor, Advanced Machine Learning (100 students), Harvard University, Fall. \\
2016 & \ind  Instructor, Machine Learning for NLP (50 students) (Rated 4.9/5), Harvard University, Spring. \\
2015 & \ind  Instructor, Artificial Intelligence (100 students), Harvard University, Fall. \\
2013 & \ind  Instructor (with Michael Collins), Natural Language Processing, Columbia University, Fall. \\
& \ind Head Teaching Assistant, Natural Language Processing , Michael Collins, Columbia University, Spring (taught on Coursera, 30,000+ registered students). \\
2012 & \ind Head Teaching Assistant, Natural Language Processing, Michael Collins, Columbia University, Fall.\\
\end{tabular}

 \bigskip
\marginhead{ {\vskip 0.3em}Patents}

% \medskip
\ind A neural attention model for abstactive summarization (Facebook). Alexander M. Rush, Sumit Chopra, Jason Weston 2017.
\medskip

\ind Techniques for discriminative dependency parsing (Google). Slav Petrov, Alexander M. Rush, 2015.
\medskip

\ind Efficient parsing with structured prediction cascades (Google). Slav Petrov, Alexander M. Rush, 2013
\medskip

%% Publications
\ind  Determining user affinity towards applications on a social networking website (Facebook., Thomas S. Whitnah, Alexander M. Rush, Ding Zhou, Ruchi Sangvhi, 2010.



\bigskip

%% Publications

\marginhead{ {\vskip 0.4em}Personal Libraries}
\bigskip

\ind \href{ paper.link } { {{paper.title}}. }
\begin{itemize}
\item {{paper.abstract}}.
\end{itemize}
\medskip





% \medskip \ind \href{ http://seq2seq-vis.io/ }{ LSTMVis / Seq2Seq-Vis }
% \begin{itemize}
% \item Collaboration with IBM to develop an interactive visualization for translation models.
% \end{itemize}



% \medskip \ind \href{ http://opennmt.net }{ OpenNMT }
% \begin{itemize}
% \item  Open-source neural machine translation system. Used in over 200 publications, and
%   deployed by major translation providers such as Systran International.
% \end{itemize}



% \medskip \ind \href{ http://nlp.seas.harvard.edu/2018/04/03/attention.html }{ The Annotated Transformer }
% \begin{itemize}
% \item  An educational line-by-Line implementation of Google's Transformer architecture.
%   Viewed over 300,000 times.
% \end{itemize}

% \medskip \ind \href{ http://nlp.seas.harvard.edu/NamedTensor }{ Named Tensor }
% \begin{itemize}
% \item  An educational view of the use of tensors in Pytorch.
%     Viewed over 50,000 times over the year.
% \end{itemize}


% 
% 
% \medskip \ind \href{ {{proj.link}} }{ {{proj.title}} }
% \begin{itemize}
% \item  {{proj.abstract}}
% \end{itemize}
% 


 \bigskip


%% Publications


%% Publications

\marginhead{ {\vskip 0.4em}Academic \newline Internships}

\bigskip

\ind Research Intern, {\sl Google Research}, 2011 -- 2013 , New York, NY.  Advisor: Slav Petrov.



\ind Research Intern, {\sl USC/ISI },  Summer 2010, Marina Del Rey, CA. Advisor: Liang Huang.

\bigskip

%% Publications

\marginhead{ {\vskip 0.4em}Industry }

\medskip
\ind  Lead Engineer (Platform Team), {\sl Facebook}, 2007 -- 2009, Palo Alto, CA.
\begin{itemize}
\item Developed compiler for Facebook Markup Language (FBML) to sanitize user content.
\item Developed system for crowd-sourced translation of Facebook user text.
\end{itemize}



% \medskip \ind \href{http://www.pydecode.org}{PyDecode} (http://www.pydecode.org)
% \begin{itemize}
% \item  An optimized toolkit for designing inference algorithms for NLP.
% \end{itemize}

% \ind  \href{http://www.declassification-engine.org}{The Declassification Engine} (http://www.declassification-engine.org)
% \begin{itemize}
% \item  Computational social science project analyzing the process of official secrecy.
% \item Awarded a Magic grant from Brown Institute for Media Innovation.
% \end{itemize}
% \medskip

\bigskip

\marginhead{ {\vskip 0.4em}Invited Talks \newline}
\medskip
\hspace{-1cm} \begin{tabular}{lp{11.5cm}}
                2023
                & \ind   Panel, NeurIPS Keynote. \\
                & \ind   Invited Talk, JHU. \\
                & \ind   Invited Talk, NYU. \\
                & \ind   Keynote Talk, MLSys 2023. \\
                & \ind   Keynote Talk, Simon Workshop on LLMs . \\
                & \ind   Invited Talk, Dagstuhl Seminar. \\
                & \ind   Invited Talk, UCSD AI Seminar. \\
                & \ind   Invited Talk, Penn NLP Seminar. \\
                & \ind   Invited Talk, Stanford NLP Seminar. \\
                2022
                & \ind   Invited Talk, SoCal NLP. \\
                & \ind   Invited Talk, LXMLS 2022. \\
                & \ind   Invited Talk, MASC 2022. \\
                & \ind   Invited Talk, Rutgers Efficient ML. \\
                & \ind   Invited Talk, Georgia Tech NLP Seminar. \\
                & \ind   Invited Talk, Pacific Research Lab NLP. \\
                2021
                & \ind   Invited Talk, NeurIPS Crowd Source ML. \\
                & \ind   Invited Talk, NeurIPS AIPLANs Workshop. \\
                & \ind   Invited Talk, Microsoft Efficient ML. \\
                & \ind   Invited Talk, Stanford SysML. \\
                & \ind   Invited Talk, Lisbon Machine Learning. \\
                 & \ind   Invited Talk, University of Cambridge. \\
                 & \ind   Invited Talk, Oracle. \\
                 & \ind   Invited Talk, UCSB. \\
                2020
                 & \ind   Invited Talk, ByteDance. \\
                 & \ind   Invited Talk, Baidu. \\
                 & \ind   Colloquium, UMass Lowell. \\
                 & \ind   Invited Talk, London Machine Learning. \\
                 & \ind   Invited Talk, UCSB. \\
                 & \ind   Invited Talk, Baidu. \\
                 & \ind   Invited Talk, Google AI. \\
                 & \ind   Invited Talk, Oracle AI. \\
                 & \ind   Invited Talk, EMNLP Structured Prediction WS. \\
                 & \ind   Invited Talk, EMNLP SustaiNLP WS. \\
              \end{tabular}

\hspace{-1cm} \begin{tabular}{lp{11.5cm}}
                2019
                & \ind   Colloquium, University of Edinburgh, Spring.\\
                & \ind   Colloquium, Tel Aviv University, Spring. \\
                 & \ind   Invited Talk, Conversational Intelligence Summer. \\
                & \ind   Invited Talk, GANocracy, MIT, Summer. \\
                & \ind   Invited Talk, OpenAI, MIT, Summer. \\
                & \ind   Invited Talk, NeuralGen Workshop NAACL Summer. \\
                & \ind   Invited Talk, Berkeley NLP, Fall.\\
                & \ind   Keynote, PyTorch Developers Conference, Fall. \\
              \end{tabular}

\hspace{-1cm} \begin{tabular}{lp{11.5cm}}
                2018
 & \ind   Invited Talk, University of Washington, Spring. \\
 & \ind   Invited Talk, Allen Institute for AI, Spring. \\
 & \ind   Invited Talk, MSR, Spring. \\
 & \ind   Keynote, American Machine Translation Association, Spring. \\
 & \ind   Invited Talk, University of Texas, Spring. \\
 & \ind   Invited Talk, University of Maryland, Spring. \\
 & \ind   Invited Talk, Georgetown, Spring. \\
 & \ind   Invited Talk, Lisbon ML Summer School, Summer. \\
 & \ind   Invited Talk, Columbia University, Fall. \\
 & \ind   Invited Talk, New York University - Text as Data, Fall. \\
 & \ind   Tutorial, EMNLP, Fall. \\
              \end{tabular}

\hspace{-1cm} \begin{tabular}{lp{11.5cm}}
2017
 & \ind   Invited Talk, Google Faculty Day, Spring. \\
 & \ind   Invited Talk, New England Machine Learning Day, Spring. \\
 & \ind   Invited Talk, Google, Spring. \\
 & \ind   Invited Talk, Berkeley CS, Spring. \\
 & \ind   Invited Talk, Notre Dame, Spring. \\
 & \ind   Colloquium, TTI-Chicago, Spring. \\
 & \ind   Invited Talk, Apple, Siri Team, Spring. \\
 & \ind   Colloquium, Samsung Global AI Forum, Fall. \\
 & \ind   Invited Talk, AMD, Fall. \\
              \end{tabular}

\hspace{-1cm} \begin{tabular}{lp{11.5cm}}
 2016
 & \ind   Invited Talk, NYU, Fall. \\
 & \ind   Invited Talk, BBN Research, Fall. \\
 & \ind   Invited Talk, Bloomberg, Fall. \\
 & \ind   Invited Talk (Speech Group), MIT, Fall. \\
 & \ind   Invited Talk, IBM Research, Fall. \\
 & \ind   Colloquium, CMU, Fall. \\
 & \ind   Invited Talk, Stanford NLP, Summer. \\
 & \ind   Invited Talk, Oracle Labs, Summer. \\
 & \ind   Invited Talk, Twitter, Summer. \\
 & \ind   Colloquium, John Hopkins University, Spring. \\
 & \ind   Colloquium, Rakuten, Spring. \\
\end{tabular}
\begin{tabular}{lp{11.5cm}}

 2014

 & \ind   Colloquium, University of Washington, Spring. \\

 & \ind   Colloquium, NYU, Spring. \\

 & \ind   Colloquium, CMU, Spring. \\

 & \ind   Colloquium, MIT, Spring. \\

 & \ind   Colloquium, Harvard, Spring. \\

 & \ind   Colloquium, TTIC, Spring. \\

 & \ind   Colloquium, University of Maryland, Spring. \\
\end{tabular}
\hspace{-1cm} \begin{tabular}{lp{11.5cm}}
 2013
 & \ind   Invited Tutorial, UMBC, October. \\

 & \ind   Invited Talk, CS and Social Science Seminar, UMass Amherst, October. \\

 & \ind   Talk, NLP Seminar, Columbia University, October. \\
 & \ind   Invited Talk, ML Seminar, UMass Amherst, October. \\

  & \ind  Invited Talk, Johnson Research Labs, NY, August. \\

  &  \ind Invited Talk, Society for Historians of American Foreign Relations, Arlington, June. \\

  & \ind  Invited Talk, Columbia University, Spring. \\

  &  \ind    Invited Talk, NLP Seminar, City University of New York, Spring. \\
 2012 & \ind   Invited Tutorial, Neural Information Processing Systems (NIPS), December. \\
 2011 & \ind  Invited Tutorial, Google Research, Mountain View, August. \\
  &  \ind Tutorial. Association of Computational Linguistic (ACL), June. \\
  & \ind  Invited Talk, ML Seminar, University of Massachusetts, Amherst, Spring. \\
 & \ind   ML Tea, MIT, January. \\
  2010 & \ind  NLP Seminar, USC/ISI, Summer. \\
 2006 &  \ind  Invited Talk, Computational Linguistics Seminar, University of Pennsylvania, November. \\
\end{tabular}
%\end{revnumerate}


%\newpage


\bigskip

%% Publications

% \marginhead{ {\vskip 0.4em}Misc}
% \medskip

% \ind General Manager of Harvard Radio Broadcasting (WHRB).

% \medskip

% \ind Member ACM Programming Team, Harvard University.

% \medskip

% \ind Experienced in C/C++, Python, Javascript, Lua/Torch, Haskell, Java, PHP, OCaml, LaTeX, HTML/CSS.

\end{document}
